\beginsong{Mer losse d'r Dom en Kölle}[by={Bläck Föös}]
\beginchorus
\ifchorded
{\nolyrics Intro: | \[E] | \[E] | \[E] | \[E] |}
\fi
Mer \[A]losse d'r Dom en Kölle, denn do jehööt hä \[E7]hin.
Wat sull di dann woanders, dat hätt doch keine \[A]Senn.
Mer losse d'r Dom in Kölle, denn do es \[A7]hä ze \[D]huss.
Un op singem \[A]ahle Platz, bliev \[E7]hä och joot \[G#]en \[A]Schuss,
\[D]un op singem \[A]ahle Platz, bliev \[E7]hä och joot en \[A]Schuss.\musicnote{slow beat}
\endchorus
\beginverse
\[A]Stell d'r für, de Kreml stünd o'm \[E7]Ebertplatz, stell d'r für, de Louvre stünd am \[A]Ring.
Do wör für die zwei doch vell ze \[E]winnich Platz, \[H7]dat wör doch e unvorstellbar \[E]Ding.
\[A]Am Jürzenich, do wör vielleich et \[E7]Pentajon, am Rothus stünd dann die Akropo\[A]lis.
Do wöss mer över haup nit, wo mer \[E]hinjonn sullt, un \[H7]daröm es dat eine janz je\[E]wess:
\endverse
\beginchorus
Mer \[A]losse d'r Dom en Kölle, denn do jehööt hä \[E7]hin.
Wat sull di dann woanders, dat hätt doch keine \[A]Senn.
Mer losse d'r Dom in Kölle, denn do es \[A7]hä ze \[D]huss.
Un op singem \[A]ahle Platz, bliev \[E7]hä och joot \[G#]en \[A]Schuss,
\[D]un op singem \[A]ahle Platz, bliev \[E7]hä och joot en \[A]Schuss.\musicnote{slow beat}
\endchorus
\beginverse
Die \[A]Ihrestross, die hieß vielleich \[E7]Sixth Avenue, oder die Nordsüd-Fahrt Brenner\[A]pass.
D'r Mont Klamott, dä heiss op eimol \[E]Zuckerhut, do \[H7]köm dat Panorama schwer in \[E]Brass.
\[A]Jetzt froch ich üch, wem domet je\[E7]holfe es, wat nötz die janze Stadtsanierung \[A]schon?
Do sull doch leever alles blieve \[E]wie et es un \[H7]mir behaale uns're schöne \[E]Dom.
\endverse
\beginchorus
Mer \[A]losse d'r Dom en Kölle, denn do jehööt hä \[E7]hin.
Wat sull di dann woanders, dat hätt doch keine \[A]Senn.
Mer losse d'r Dom in Kölle, denn do es \[A7]hä ze \[D]huss.
Un op singem \[A]ahle Platz, bliev \[E7]hä och joot \[G#]en \[A]Schuss,
\[D]un op singem \[A]ahle Platz, bliev \[E7]hä och joot en \[A]Schuss.\[E7]\[A]
\endchorus
\endsong
