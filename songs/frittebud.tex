
\beginsong{Frittebud}[by={Gero Kuntermann, Peter Gammersbach}]
\beginverse
\ifchorded
{\nolyrics Intro: | \[D] | \[A7] | \[A7] | \[D] |}
\ifdefined\isaccordeon
\begin{center}
\begin{lilypond}
\relative c'' {
  \key d \major a8 fis16 b16 ~ b4 a8 fis16 b16 ~ b4 a8 fis b fis a gis16 g16 ~ g4 g8 e16 a16 ~ a4 g8 e16 a16 ~ a8 fis16 g a a g g fis fis e e d2
}
\end{lilypond}
\end{center}
\fi
\fi
\[D]Mama hät hück jar \[A]kein Lust ze koche,
\[A]Mama will hück schön na'm \[D]Städtsche john.
\[Bm]Endlich neue \[A]Schuhe kaufen
\[A]ja, dat wollt et \[D]lange schon.
\[D]Papa der soll hück dat \[A]Middach mache,
\[A]doch he kann nur \[D]Spiejelei.
\[Bm]Spiegelei is  \[A]anjebrannt
\[A]da hilft nur en Fr\[D]itiererei.
\endverse
\beginchorus
2x
\[D]Frittebud, Frittebud,
Fritte schmecke im\[A]ma jut.
Ruut und wiess, ming Paradies,
für Fritte bin ich ni\[D]emals fies.
\endchorus
\beginverse
\ifchorded
{\nolyrics Intro: | \[D] | \[A7] | \[A7] | \[D] |}
\fi
Mama will jetz auch emol verreise.
Mama meint dat muss wohl möschlisch sin.
"Jung!" säht se, "Du bis jetz 33, 
kriest dat ohne misch ens hin!"
"Oh Mama, wie sull isch misch ernähre?
Isch werd sicher  janz dünn un malad!"
Doch die Mama wör nit minge Mama,
hät se nit die Lösung längst  parat:
\endverse
\beginchorus
2x
Frittebud, Frittebud,
Fritte schmecke im ma jut.
Ruut und wiess, ming Paradies,
für Fritte bin ich niemals fies.
\endchorus
\beginverse
\ifchorded
{\nolyrics Intro: | \[D] | \[A7] | \[A7] | \[D] |}
\fi
Frieda mäht am Frieseplatz de Fritte,
Frieda is en Fritten-Fritier-Frau.
Frieda is so scharf wie Frittepfeffer,
kennt minge Wünsche janz jenau.
Fridach werd ich Frieda froje.
Fridach säht et "Jo", et is jewiss.
Endlisch stonn mir zwei fürm Traualtar.
Ihr wisst schon, wo dr Huhzick is:
\endverse
\beginchorus
2x
Frittebud, Frittebud,
Fritte schmecke im ma jut.
Ruut und wiess, ming Paradies,
für Fritte bin ich niemals fies.
\endchorus
 
\endsong
