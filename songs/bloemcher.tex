
\beginsong{Mer schenke der Ahl e paar Blömcher}[by={Lotti Krekel}]
\beginverse
Em janze \[C]Veedel es die ahl Frau Schmitz \[G7]bekannt.
Se weed vun \[Dm]allen nur et Sch\[G7]mitze Bell jen\[C]annt.
Die hätt nit viel, es nit besonders \[G7]rich.
Un hätt noch \[Dm]lang nit jede \[G7]Meddaach Fleisch om D\[C]esch.
Nur \[F]ein Deil jit et, wo se Freud' dran \[C]hätt.
Dat sinn die B\[D7]lömscher op ihrem Finsterb\[G]rett.
\endverse
\beginchorus
Mer sch\[C]enke dä Ahl en paar Blömscher
e paar B\[G7]lömscher für ihr Finster\[C]brett.
Mer \[C]schenke ihr e paar Bl\[G]ömscher,
denn die \[G7]ahl Frau Schmitz, die es esu \[C]nett.
\endchorus
\beginverse
Un klopp och öfters ens 'ne Ärme an ihr Dür.
Dat se janix jitt, ich jläuv dat kütt nit vür.
Un se es och nit rich, es keine Milljonär.
Jet zo verschenke, dat fällt ihr jar nit schwer.
Un sinn et nur zehn Penning un nit mih.
Dovür hät se ävver usere Sympathie.
\endverse
\beginchorus
Mir schenke dä Ahl en paar Blömscher
e paar Blömscher für ihr Finsterbrett.
Mir schenke ihr e paar Blömscher,
denn die ahl Frau Schmitz, die es esu nett.
\endchorus
\beginverse
Un es die ahl Frau Schmitz ens einmol nit mih do.
Dann deit dat manchem wih, dat es doch klor.
Un wor se och nit rich, hatt net besonders vill.
Su wor se doch für uns all et Schmitze Bell.
Un wenn für sie och längs kein Blom mih blöht.
Dann singe mer für sie noch ens dat Leed.
\endverse
\beginchorus
Mir schenke dä Ahl en paar Blömscher
e paar Blömscher für ihr Finsterbrett.
Mir schenke ihr e paar Blömscher,
denn die ahl Frau Schmitz, die es esu nett.
\endchorus
\beginchorus
Mir schenke dä Ahl en paar Blömscher
e paar Blömscher für ihr Finsterbrett.
Mir schenke ihr e paar Blömscher,
denn die ahl Frau Schmitz, die es esu nett.
\endchorus

\endsong